\documentclass[thesis]{cluu}
\usepackage{tabularray}
\usepackage{pythonhighlight}

\usepackage[style=cluu]{biblatex}
\addbibresource{pal.bib}

\begin{document}
\course{Introduction to wordplay, 7.5hp}
\author{Ellen Kaka}
\title{Palindromes}
\supervisor{Linda Adnil, Ytisrevinu University}
\subtitle{Having it both ways}

\maketitle

\begin{abstract}
  The concept of \emph{palindromes} is introduced, and some English
  palindromes are analyzed.
\end{abstract}

\tableofcontents

\addchap{Preface}
I want to thank Donald Knuth for making \TeX, without which
I wouldn't have written this.

% by using \addchap instead of \chapter this preface isn't numbered.

\chapter{Introduction}

My assignment was to find and analyze some palindromes.
I will summarize my findings in Chapter~\ref{chap:results}.

\chapter{Theory}

\section{Definition}

A \emph{palindrome} is a word or phrase which reads the same forwards and
backwards. For a phrase interpunction is ignored.

\section{Math}
The palindromic density of an infinite word \( w \) over an alphabet
\( A \) is defined to be zero if only finitely many prefixes are
palindromes; otherwise, letting the palindromic prefixes be of lengths
\( n_k \) for \( k=1,2,\dots \) we define the density to be  

\begin{equation}
  d_P(w) = \left( { \limsup_{k \rightarrow \infty} \frac{n_{k+1}}{n_k} } \right)^{-1}
  \label{eq:density}
\end{equation}
which maybe makes sense.
The \emph{Fibonacci word}\note{See
  \url{https://en.wikipedia.org/wiki/Fibonacci_word}}
  has the density \( 1/\phi \), as shown by
  \textcite[443]{adamczewski10}.

\chapter{The words and the people}

\section{Some examples}
\label{sec:examples}

Some good ones are listed at
\url{https://en.wikipedia.org/wiki/Palindrome}. Some of the words in
English are \emph{civic, radar, kayak} and \emph{racecar}.

Here are some phrases in English:
\begin{itemize}
\item A man, a plan, a canal -- Panama
\item Mr. Owl ate my metal worm
\item Do geese see God?
\item Was it a car or a cat I saw?
\item Rats live on no evil star
% \item Live on time, emit no evil
% \item Step on no pets
\end{itemize}

In Yreka, California there has been an establishment with a sign
reading \q{Yreka Bakery} which was observed in a children's magazine
in 1866 \parencite[30]{eckler01}.

The longest palindromic \emph{word} in the \emph{Oxford English Dictionary}
is the onomatopoeic \emph{tattarrattat}, coined by
\textcite{joyce22} for a knock on the door.

More words can be found with the code in Appendix~\ref{chap:code}.

\section{Palindromists}

For English John Taylor coined a palindromic phrase in 1614 which
generally is considered the first English-language palindrome
sentence. The pseudonym J.T.R. coined a famous palindrome in 1848,
but it is unknown who that person is.

Table~\ref{tab:notables} shows some notable palindromists, sorted by
year of birth.
\Textcite{eckler91} gives more information about Leigh Mercer.

\begin{table}
  \centering
  \caption{Notable palindromists}
  \label{tab:notables}
  \begin{tblr}{
      colspec = {lc},
      hlines = {white},
      row{odd} = {blue!8!white}, row{even} = {blue!12!white},
    }
    Velimir Khlebnikov & 1885--1922 \\
    Leigh Mercer & 1893--1977 \\
    J. A. Lindon & c. 1914--1979 \\
    Howard W. Bergerson & 1922--2011 \\
    Hugo Brandt Corstius & 1935--2014 \\
    Georges Perec & 1936--1982 \\
    Dmitry Avaliani & 1938--2003 \\
  \end{tblr}
\end{table}

\chapter{Results}
\label{chap:results}

I examined a list of given names, and found a few there:
\emph{Anna, Hannah} and \emph{Otto}.
Also I found Equation~\ref{eq:density} at Wikipedia, but had no use for it.
But don't miss the examples in Section~\ref{sec:examples}, or my
program in Appendix~\ref{chap:code}!

\appendix
\chapter{The pal program}
\label{chap:code}

\inputpython{pal.py}{1}{500}

\printbibliography

\end{document}
